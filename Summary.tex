\documentclass[aps,pra,preprint,groupedaddress]{revtex4-1}
\usepackage{graphicx}
\usepackage{amsmath}

\begin{document}

\title{Testing the 2D Classical Model of Phase-Dependent Ionization}

\author{Eric Magnuson}
\email[]{edm5gb@virginia.edu}
\affiliation{University of Virginia, Department of Physics}
\date{\today}

\begin{abstract}
Prior works have discussed the one dimensional model for Rydberg-MW interactions in several works. It is useful to turn the mathematical description into a computational model. Prior work used a 1D single orbit model. In this work, I will test and apply a 2D Model.
\end{abstract}

\section{\label{sec:State} Explicit Statement of Model}

\subsection{\label{sec:AU} Atomic Units}

Calculations are simplified by using atomic units. Useful atomic units, found in Rydberg Atoms by Tom Gallagher \emph{(CITE)}

\begin{align*}
\text{Energy} & & E_{AU} & & & 4.35974417 \cdot 10^{-18} ~ \text{J} \\
\text{Time} & & t_{AU} & & & 2.418884326505 \cdot 10^{-17} ~ \text{s} \\
\text{Field} & & f_{AU} & & & 5.14220652 \cdot 10^{11} ~ \text{V/m}
\end{align*}

After calculations, it's useful to transform back to lab units.

\begin{align*}
\text{1 GHz} & = 1.51983 \cdot 10^{-7} ~ E_{AU} \\
\text{1 ns} & = 4.13414 \cdot 10^7 ~ t_{AU} \\
\text{1 mV/cm} & = 1.94469 \cdot 10^{-13} ~ f_{AU}
\end{align*}

\subsection{\label{sec:IC} Initial Conditions}

Experimental parameters provide the initial energy $E_0$ and angular momentum $L_0$. This is further constrained by the choice to start at the periapsis of a Keplerean trajectory.

The system of equations for the initial conditions are
\begin{align*}
E_0 & = \frac{1}{2} v_0^2 - \frac{1}{r_0} & L_0 & = r_0 \cdot v_0
\end{align*}
which has two solutions representing the peri- and apoapsis. Excitation from the 3D state suggests periapsis initial conditions.
\begin{align*}
r_0 & = \frac{-1 + \sqrt{1 + 2 E_0 L_0^2}}{2 E_0} & v_0 & = \frac{1 + \sqrt{1 + 2 E_0 L_0^2}}{L_0}
\end{align*}
These initial conditions need to be mapped onto the angle of the Laplace-Runge-Lenz (LRL) vector in the x-z plane, and the selection of the angular momentum in the $\pm \hat{y}$ direction. This is implemented by starting with $\theta_{LRL} = 0$ and $\hat{L} = \pm \hat{y}$ and then rotating $\vec{r}_0, \vec{v}_0$.
\begin{align*}
& & x(0) & = \sin{\theta_{LRL}} \cdot r_0 & z(0) & = -\cos{\theta_{LRL}} \cdot r_0 \\
\hat{L} & = \pm \hat{y} & \dot{x}(0) & = \pm \cos{\theta_{LRL}} \cdot v_0 & \dot{z}(0) & = \pm \sin{\theta_{LRL}} \cdot v_0
\end{align*}

\subsection{\label{sec:EoM} Equations of Motion}

The equations of motion must include the Coulomb potential, applied pulsed field, and microwave field. The vector statement of the equations of motion, in atomic units, is

\begin{align*}
\ddot{\vec{r}} & = F_{coul}(\vec{r}) - \vec{E}_{P}(t) - \vec{E}_{MW}(t) \\
 & = -\frac{1}{r^2} \cdot \hat{r} - \Phi_P(t) \cdot E_{p} \cdot \hat{z} - \Phi_{MW}(t) \cdot E_{mw} \sin{(\omega t + \phi_0)} \cdot \hat{z} 
\end{align*}
where the $\Phi_P$ and $\Phi_{MW}$ are envelope functions describing the square wave turning off the pulsed field and the exponential ringdown of the MW field.

\begin{align*}
\Phi_P(t \leq t_{off}) & = 1 & \Phi_{MW}(t \leq t_{off}) & = 1 \\
\Phi_P(t > t_{off}) & = 0 & \Phi_{MW}(t > t_{off}) & = e^{-(t-t_{off})/\tau_{MW}}
\end{align*}

For computation, these vector equations need to be expressed in terms of Cartesian coordinates \{x,y,z\}. Including the initial conditions from Sec.~\ref{sec:IC}, the system of ODEs is:
\begin{align*}
x(0) & = \sin{\theta_{LRL}} \cdot r_0 & z(0) & = -\cos{\theta_{LRL}} \cdot r_0 \\
\dot{x}(0) & = \pm \cos{\theta_{LRL}} \cdot v_0 & \dot{z}(0) & = \pm \sin{\theta_{LRL}} \cdot v_0 \quad \quad \hat{L} = \pm \hat{y} \\
\ddot{x} & = -\frac{x}{(x^2 + z^2)^{3/2}} & \ddot{z} & = -\frac{z}{(x^2 + z^2)^{3/2}} - \Phi_P(t) \cdot E_P - \Phi_{MW}(t) \cdot E_{MW} \cdot \sin{(\omega t + \phi_0)}
\end{align*}

\section{\label{sec:Tests} Tests}

\subsection{\label{sec:ThL} $\Theta_{LRL}$ and $\hat{L}$}

Fig.~\ref{fig:ThL} shows changing the initial conditions for $\hat{L}$ and $\theta_{LRL}$ produces the desired orbits.

\begin{figure}
	\includegraphics[width=0.8\textwidth]{OrbitAlignment}
	\caption{Orbits at 4 settings of $\Theta_{LRL}$ and $\hat{L}$}
	\label{fig:ThL}
\end{figure}

\subsection{\label{sec:EvP} $\Delta$ E vs $\phi_0$}

To measure $\Delta E(\phi_0)$, I ran the model in zero static field for 30 MW cycles, and took an average of the orbital energy over cycles 10 through 20. This showed a $<1 ~ GHz$ disagreement with running the simulation for 100 ns and averaging over the final 10 cycles.
\begin{equation*}
E_f = \frac{1}{10T_{MW}} \int_{10T_{MW}}^{20T_{MW}} \left(\frac{v(t)^2}{2} - \frac{1}{r(t)}\right) \cdot dt
\end{equation*}

Carrat, 2015 shows the 1-D analytical theory predicts a maximum energy transfer of
\begin{equation*}
\Delta E_{\pi/6} = \frac{3}{2} \frac{E_{MW}}{\omega^{2/3}}
\end{equation*}
I've tested this on three microwave field values of $E_{MW} = 2, ~ 4, ~ \text{and} ~ 6 ~ V/cm$ at initial energies of $E_0 = 0, ~ 100, ~ \text{and} ~ 200 ~ GHz$. The test at the ionization limit agrees well, while those further from the limit are less accurate.

Beyond maximum energy transfer, the model shows very good agreement to a sinusoidal dependence.
\begin{align*}
\Delta E(\phi_0) & = A \cdot \cos{(\phi_0 - \Delta \phi)} & A & = \frac{3}{2} \frac{E_{MW}}{\omega^{2/3}} & \Delta \phi & = \frac{\pi}{6}
\end{align*}
where $\phi_0$ refers to the excitation phase of the MW field
\begin{equation*}
\vec{E}_{MW}(t) = E_{MW} \sin{(\omega t + \phi_0)} \cdot \hat{z}
\end{equation*}
For $E_0 = 0 ~ GHz$, Table~\ref{tab:EvP} shows the parameters predicted by theory and the parameters fit to the computational model, showing good agreement. For the same conditions, Fig.~\ref{fig:EvP} shows the computed final energies in color traced with the theory predictions in black.

\begin{table}
	\caption{\label{tab:EvP} Amplitudes and phases predicted by theory ($A_{pred}, \Delta \phi_{pred}$) and produced by our computational model ($A_{comp}, \Delta\phi_{comp}$) for three MW field strengths. Each calculation is at an initial energy $E_0 = 0 ~ GHz$ and microwave frequency $f_{MW} = 15.9 ~ GHz$.}
	\begin{ruledtabular}
	\begin{tabular}{c|cc|cc}
	$E_{MW}$ (V/cm) & $A_{pred}$ (GHz) & $A_{comp}$ (GHz) & $\Delta \phi_{pred}$ (rad.) & $\Delta \phi_{comp}$ (rad.) \\ \hline
	2.0 & 21.317 & 21.172 & 0.524 & 0.523 \\
	4.0 & 42.633 & 42.349 & 0.524 & 0.523 \\
	6.0 & 63.949 & 63.543 & 0.524 & 0.523
	\end{tabular}
	\end{ruledtabular}
\end{table}

\begin{figure}
	\includegraphics[width=0.8\textwidth]{EvP}
	\caption{\label{fig:EvP} Total energy gained by the electron $(\Delta E)$ from the MW field as it leaves the core at a particular MW phase $(\phi_0)$. The results of the computational model are points in color, and the theory prediction of $\Delta E = 3/2 \cdot E_{MW}/\omega^{3/2} \cdot \cos{(\phi_0 + \Delta\phi)}$ is traced as a solid black line, showing good agreement.}
\end{figure}

The case for a slingshotting electron is not quite double that for an electron leaving the core. If the closest approach occurs at $\phi_0$, then the energy gain arriving can be treated as energy lost by an electron leaving at $\phi_0$ in an electric field evolving backwards in time.
\begin{align*}
E_{MW}(-t,\phi_0) & = A \sin{(-\omega t + \phi_0)} \\
 & = -A\sin({\omega t - \phi_0)} \\
 & = -E_{MW}(t, -\phi_0)
\end{align*}
\begin{align*}
\Delta E_{slingshot} & = \Delta E(\phi_0) - -\Delta E(-\phi_0) \\
 & = \Delta E(\phi_0) + \Delta E(-\phi_0) \\
 & = A \cos{(\phi_0 - \Delta \phi)} + A \cos{(-\phi_0 - \Delta \phi)} \\
 & = 2 A \cos{\phi_0} \cos{\Delta \phi}
\end{align*}
For $\Delta \phi = \pi/6$, this suggests a maximum "slingshot" exchange of
\begin{equation*}
\Delta E_{slingshot,max} = \sqrt{3} A = \sqrt{3} \, \frac{3}{2} \frac{E_{MW}}{\omega^{2/3}}
\end{equation*}

\section{\label{sec:static} Uphill and Downhill Electrons in Static Fields}

\subsection{\label{sec:DHstatic} Downhill Electrons in Static Fields}

\begin{figure}
\includegraphics[width=\textwidth]{DIL}
\caption{\label{fig:DIL} The dynamics of "downhill" electrons are dominated by the depressed ionization limit in the pulsed field. \textbf{(a)} Total potential seen by an electron in a static field. "Downhill" electrons see a potential barrier of $V_{DIL} = -2\sqrt{E}$. \textbf{(b)} DIL as a function of applied Field.}
\end{figure}

\end{document}